
\begin{frame}[fragile]
\frametitle{Klassen}
\begin{itemize}
\item Sehr grobe Darstellung von Klassen für die Praktische Verwendung.
\item Klassen erlauben einem C++ Programmierer eigene Datentypen zu definieren.
\item Es gibt unter anderem Klassenvariablen und Klassenmethoden.
\item Klassenvariablen sind die Variablen aus denen der neue Datentyp zusammengesetzt ist.
\item Können elementare Datentypen wie int sein oder aber auch andere Klassen.
\item Klassenmethoden sind Funktionen mit denen Klassenvariablen manipuliert werden können.
\end{itemize}
\end{frame}


\begin{frame}[fragile,allowframebreaks]
\frametitle{Klassen}
\lstinputlisting[basicstyle=\ttfamily\scriptsize,numbers=left,
numberstyle=\tiny, numbersep=5pt]{../examples/progkurs/classes.cc}
\end{frame}



\begin{frame}[fragile]
\frametitle{Klassen III}
\begin{itemize}
\item Klassenvariablen können grundsätzlich nur durch Klassenmethoden modifiziert oder gelesen werden!
\item Ausnahme: Variablen die als "public" deklariert wurden.
\item Verschiedene Variablen einer Klasse (wie eben V und W) teilen sich ihre Untervariablen nicht.
\item Zugriff auf Klassenelemente erfolgt über den . Operator: Also wie eben V.getNorm(), V.x, W.x
\end{itemize}
\end{frame}

\begin{frame}[fragile]
\frametitle{Klassenbibliotheken}
\begin{itemize}
\item Klassen muss man (zum Glück) oft nicht selbst programmieren.
\item Es gibt bereits fertige Klassenbibliotheken.
\item Im folgenden beschäftigen wir uns mit der HDNum-Bibiliothek für den Numerik-Kurs
\end{itemize}
\end{frame}

\begin{frame}[fragile]
\frametitle{HDNUM}
\begin{itemize}
\item C++ kennt keine Matrizen, Vektoren, Polynome, \ldots
\item Wir haben C++ erweitert um die \textbf{Heidelberg Educational
  Numerics Library}, kurz \textbf{HDNum}.
\item Alle in der Vorlesung behandelten Beispiele sind dort
  enthalten.
\item Dieser Programmierkurs ist auch Teil von HDNUM
\end{itemize}
\end{frame}

\begin{frame}[fragile]
\frametitle{HDNUM}
\begin{itemize}
\item HDNum realisiert Vektoren, Matrizen usw. als Klassen-Templates
\item Klassen-Templates analog zu Funktions-Templates
\item d.h. Vektoren, Matrizen mit Elementen verschiedener Datentypen
\item Klassenmethoden für Lineare Algebra, z.b. Matrizenmultiplikation, Skalarprodukte
\item Einbinden über \#include "hdnum.hh"
\item + Angabe des Verzeichnisses der Datei hdnum.hh über -I<Verzeichnis> beim kompilieren
\item Detaillierte Anleitung im repository unter hdnum/tutorial
\end{itemize}
\end{frame}

\begin{frame}[fragile]
\frametitle{HDNUM Vektoren}
\lstinputlisting[basicstyle=\ttfamily\scriptsize,numbers=left,
numberstyle=\tiny, numbersep=5pt]{../examples/progkurs/vektoren.cc}
\end{frame}

\begin{frame}[fragile]
\frametitle{HDNUM Matrizen}
\lstinputlisting[basicstyle=\ttfamily\scriptsize,numbers=left,
numberstyle=\tiny, numbersep=5pt]{../examples/progkurs/matrizen.cc}
\end{frame}

\begin{frame}[fragile]
\frametitle{HDNUM Ausgabe}
\lstinputlisting[basicstyle=\ttfamily\scriptsize,numbers=left,
numberstyle=\tiny, numbersep=5pt]{../examples/progkurs/ausgabe.cc}
\end{frame}


\begin{frame}[fragile]
\frametitle{Beispiel: Gram-Schmidt-Orthogonalisierung mit HDNUM}
\begin{itemize}
\item Gegeben: n (linear unabhängige) Vektoren
\item Gesucht: n orthogonale Vektoren, die den selben Unterraum aufspannen
\item Projektionen der anderen Vektoren werden raussubtrahiert
\item Entsprechend der Formel:


\end{itemize}
\large
\centering
\vspace{0.5cm}
\hspace{2cm}
$ w_{j}=v_{j}-\sum_{i=1}^{j-1} \frac{\left\langle w_{i}, v_{j}\right\rangle}{\left\langle w_{i}, w_{i}\right\rangle} w_{i}$
\end{frame}


\begin{frame}[fragile,allowframebreaks]
\frametitle{Gram-Schmidt Orthogonalisierung}
\lstinputlisting[basicstyle=\ttfamily\scriptsize,numbers=left,
numberstyle=\tiny, numbersep=5pt]{../examples/progkurs/gs.cc}
\end{frame}


\begin{frame}[fragile]
\frametitle{Debugging}
\begin{itemize}
\item Was mache ich wenn mein Programm nicht läuft?
\item Grundsätzliches:
\begin{enumerate}
\item Fehlermeldung lesen!
\item Fehlermeldung bei google eingeben (Copy-Paste) falls unklar.
\item Falls ihr gar nicht weiter kommt: Kommilitonen/Tutoren um Hilfe bitten.
\end{enumerate}
\end{itemize}
\end{frame}

\begin{frame}[fragile]
\frametitle{Debugging II}
\begin{itemize}
\item Was mache ich wenn mein Programm läuft, aber nicht das macht was ich erwarte?
\begin{enumerate}
\item Geht das Programm nochmal Zeile für Zeile durch.
\item Schaut euch den Wert eurer Variablen mit std::cout an.
\item Insbesondere Variablen in If-Statements oder Schleifen.
\item Falls möglich, Ergebnis plotten. (grundsätzlich sinnvoll)
\item Falls ihr gar nicht weiter kommt: Kommilitonen/Tutoren um Hilfe bitten.
\end{enumerate}
\end{itemize}
\end{frame}

\begin{frame}[fragile]
\frametitle{Debugging III}
\begin{itemize}
\item Typische Fehler
\begin{enumerate}
\item Semikolon vergessen
\item Klammer vergessen
\item Gro\ss- und Kleinschreibung
\item Sonstige Tippfehler
\item Variable nicht deklariert/initialisiert
\item Falscher Datentyp in Funktion eingesetzt
\item und vieles mehr...
\end{enumerate}
\end{itemize}
\end{frame}

