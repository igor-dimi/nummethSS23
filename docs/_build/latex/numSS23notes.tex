%% Generated by Sphinx.
\def\sphinxdocclass{jupyterBook}
\documentclass[a4paper,10pt,english]{jupyterBook}
\ifdefined\pdfpxdimen
   \let\sphinxpxdimen\pdfpxdimen\else\newdimen\sphinxpxdimen
\fi \sphinxpxdimen=.75bp\relax
\ifdefined\pdfimageresolution
    \pdfimageresolution= \numexpr \dimexpr1in\relax/\sphinxpxdimen\relax
\fi
%% let collapsible pdf bookmarks panel have high depth per default
\PassOptionsToPackage{bookmarksdepth=5}{hyperref}
%% turn off hyperref patch of \index as sphinx.xdy xindy module takes care of
%% suitable \hyperpage mark-up, working around hyperref-xindy incompatibility
\PassOptionsToPackage{hyperindex=false}{hyperref}
%% memoir class requires extra handling
\makeatletter\@ifclassloaded{memoir}
{\ifdefined\memhyperindexfalse\memhyperindexfalse\fi}{}\makeatother

\PassOptionsToPackage{warn}{textcomp}

\catcode`^^^^00a0\active\protected\def^^^^00a0{\leavevmode\nobreak\ }
\usepackage{cmap}
\usepackage{fontspec}
\defaultfontfeatures[\rmfamily,\sffamily,\ttfamily]{}
\usepackage{amsmath,amssymb,amstext}
\usepackage{polyglossia}
\setmainlanguage{english}



\setmainfont{FreeSerif}[
  Extension      = .otf,
  UprightFont    = *,
  ItalicFont     = *Italic,
  BoldFont       = *Bold,
  BoldItalicFont = *BoldItalic
]
\setsansfont{FreeSans}[
  Extension      = .otf,
  UprightFont    = *,
  ItalicFont     = *Oblique,
  BoldFont       = *Bold,
  BoldItalicFont = *BoldOblique,
]
\setmonofont{FreeMono}[
  Extension      = .otf,
  UprightFont    = *,
  ItalicFont     = *Oblique,
  BoldFont       = *Bold,
  BoldItalicFont = *BoldOblique,
]



\usepackage[Bjarne]{fncychap}
\usepackage[,numfigreset=1,mathnumfig]{sphinx}

\fvset{fontsize=\small}
\usepackage{geometry}


% Include hyperref last.
\usepackage{hyperref}
% Fix anchor placement for figures with captions.
\usepackage{hypcap}% it must be loaded after hyperref.
% Set up styles of URL: it should be placed after hyperref.
\urlstyle{same}

\addto\captionsenglish{\renewcommand{\contentsname}{Programming Tutorial}}

\usepackage{sphinxmessages}



        % Start of preamble defined in sphinx-jupyterbook-latex %
         \usepackage[Latin,Greek]{ucharclasses}
        \usepackage{unicode-math}
        % fixing title of the toc
        \addto\captionsenglish{\renewcommand{\contentsname}{Contents}}
        \hypersetup{
            pdfencoding=auto,
            psdextra
        }
        % End of preamble defined in sphinx-jupyterbook-latex %
        

\title{Introduction to Numerical Methods SS 23 Lecture Notes}
\date{Apr 26, 2023}
\release{}
\author{Igor Dimitrov}
\newcommand{\sphinxlogo}{\vbox{}}
\renewcommand{\releasename}{}
\makeindex
\begin{document}

\pagestyle{empty}
\sphinxmaketitle
\pagestyle{plain}
\sphinxtableofcontents
\pagestyle{normal}
\phantomsection\label{\detokenize{text/intro::doc}}


\sphinxAtStartPar
\sphinxhref{https://conan.iwr.uni-heidelberg.de/teaching/numerik0\_ss2023/}{\sphinxstyleemphasis{Introduction to Numerical Methods}} Uni Heidelberg summer semester 23 Lecture Notes.

\sphinxstepscope


\part{Programming Tutorial}

\sphinxstepscope


\chapter{Introduction}
\label{\detokenize{text/progtut/intro:introduction}}\label{\detokenize{text/progtut/intro::doc}}
\sphinxAtStartPar
C++ programming tutorial notes

\sphinxAtStartPar
Some links:
\begin{itemize}
\item {} 
\sphinxAtStartPar
hedgedoc \sphinxhref{https://hedgedoc.mathphys.info}{link}

\item {} 
\sphinxAtStartPar
git repo for hdNUM \sphinxhref{https://parcomp-git.iwr.uni-heidelberg.de/Teaching/hdnum.git}{likn}

\end{itemize}

\sphinxstepscope


\chapter{Hello World}
\label{\detokenize{text/progtut/hello:hello-world}}\label{\detokenize{text/progtut/hello::doc}}
\sphinxAtStartPar
Libraries are included with the \sphinxcode{\sphinxupquote{\#include}} directive.
\begin{itemize}
\item {} 
\sphinxAtStartPar
\sphinxcode{\sphinxupquote{<iostream>}}: standard library for standard input/output (keyboard and screen)

\item {} 
\sphinxAtStartPar
\sphinxcode{\sphinxupquote{<vector>}}: library for dynamic arrays, i.e. arrays whose size changes during run\sphinxhyphen{}time. Implemented as amortized arrays (rather than linked lists, contrary to what one would naively assume).

\item {} 
\sphinxAtStartPar
\sphinxcode{\sphinxupquote{"hdnum.hh"}}: header file to include the custoim \sphinxcode{\sphinxupquote{hdnum}} library.

\end{itemize}

\begin{sphinxadmonition}{note}{Note:}
\sphinxAtStartPar
standard libraries are enclosed with angle brackets, progammer\sphinxhyphen{}defined header files are enclosed with double quotes
\end{sphinxadmonition}

\begin{sphinxuseclass}{cell}\begin{sphinxVerbatimInput}

\begin{sphinxuseclass}{cell_input}
\begin{sphinxVerbatim}[commandchars=\\\{\}]
\PYG{c+cp}{\PYGZsh{}}\PYG{c+cp}{include}\PYG{+w}{ }\PYG{c+cpf}{\PYGZlt{}iostream\PYGZgt{}}
\PYG{c+cp}{\PYGZsh{}}\PYG{c+cp}{include}\PYG{+w}{ }\PYG{c+cpf}{\PYGZlt{}vector\PYGZgt{}}
\PYG{c+cp}{\PYGZsh{}}\PYG{c+cp}{include}\PYG{+w}{ }\PYG{c+cpf}{\PYGZdq{}../../../hdnum/hdnum.hh\PYGZdq{}}
\end{sphinxVerbatim}

\end{sphinxuseclass}\end{sphinxVerbatimInput}

\end{sphinxuseclass}
\sphinxAtStartPar
and print out some messages to standard outpout:

\begin{sphinxuseclass}{cell}\begin{sphinxVerbatimInput}

\begin{sphinxuseclass}{cell_input}
\begin{sphinxVerbatim}[commandchars=\\\{\}]
\PYG{n}{std}\PYG{o}{:}\PYG{o}{:}\PYG{n}{cout}\PYG{+w}{ }\PYG{o}{\PYGZlt{}}\PYG{o}{\PYGZlt{}}\PYG{+w}{ }\PYG{l+s}{\PYGZdq{}}\PYG{l+s}{hello}\PYG{l+s}{\PYGZdq{}}\PYG{+w}{ }\PYG{o}{\PYGZlt{}}\PYG{o}{\PYGZlt{}}\PYG{+w}{ }\PYG{n}{std}\PYG{o}{:}\PYG{o}{:}\PYG{n}{endl}\PYG{p}{;}
\PYG{n}{std}\PYG{o}{:}\PYG{o}{:}\PYG{n}{cout}\PYG{+w}{ }\PYG{o}{\PYGZlt{}}\PYG{o}{\PYGZlt{}}\PYG{+w}{ }\PYG{l+s}{\PYGZdq{}}\PYG{l+s}{1 + 1 = }\PYG{l+s}{\PYGZdq{}}\PYG{+w}{ }\PYG{o}{\PYGZlt{}}\PYG{o}{\PYGZlt{}}\PYG{+w}{ }\PYG{l+m+mi}{1}\PYG{+w}{ }\PYG{o}{+}\PYG{+w}{  }\PYG{l+m+mi}{1}\PYG{+w}{ }\PYG{o}{\PYGZlt{}}\PYG{o}{\PYGZlt{}}\PYG{+w}{ }\PYG{n}{std}\PYG{o}{:}\PYG{o}{:}\PYG{n}{endl}\PYG{p}{;}
\end{sphinxVerbatim}

\end{sphinxuseclass}\end{sphinxVerbatimInput}
\begin{sphinxVerbatimOutput}

\begin{sphinxuseclass}{cell_output}
\begin{sphinxVerbatim}[commandchars=\\\{\}]
hello
\end{sphinxVerbatim}

\begin{sphinxVerbatim}[commandchars=\\\{\}]
1 + 1 = 2
\end{sphinxVerbatim}

\end{sphinxuseclass}\end{sphinxVerbatimOutput}

\end{sphinxuseclass}
\sphinxAtStartPar
Full program text:

\begin{sphinxVerbatim}[commandchars=\\\{\}]
\PYG{c+cp}{\PYGZsh{}}\PYG{c+cp}{include}\PYG{+w}{ }\PYG{c+cpf}{\PYGZlt{}iostream\PYGZgt{}}
\PYG{c+cp}{\PYGZsh{}}\PYG{c+cp}{include}\PYG{+w}{ }\PYG{c+cpf}{\PYGZlt{}vector\PYGZgt{}}
\PYG{c+cp}{\PYGZsh{}}\PYG{c+cp}{include}\PYG{+w}{ }\PYG{c+cpf}{\PYGZdq{}../../../hdnum/hdnum.hh\PYGZdq{}}

\PYG{k+kt}{int}\PYG{+w}{ }\PYG{n+nf}{main}\PYG{p}{(}\PYG{p}{)}
\PYG{p}{\PYGZob{}}
\PYG{+w}{    }\PYG{n}{std}\PYG{o}{:}\PYG{o}{:}\PYG{n}{cout}\PYG{+w}{ }\PYG{o}{\PYGZlt{}}\PYG{o}{\PYGZlt{}}\PYG{+w}{ }\PYG{l+s}{\PYGZdq{}}\PYG{l+s}{hello}\PYG{l+s}{\PYGZdq{}}\PYG{+w}{ }\PYG{o}{\PYGZlt{}}\PYG{o}{\PYGZlt{}}\PYG{+w}{ }\PYG{n}{std}\PYG{o}{:}\PYG{o}{:}\PYG{n}{endl}\PYG{p}{;}
\PYG{+w}{    }\PYG{n}{std}\PYG{o}{:}\PYG{o}{:}\PYG{n}{cout}\PYG{+w}{ }\PYG{o}{\PYGZlt{}}\PYG{o}{\PYGZlt{}}\PYG{+w}{ }\PYG{l+s}{\PYGZdq{}}\PYG{l+s}{1 + 1}\PYG{l+s}{\PYGZdq{}}\PYG{+w}{ }\PYG{o}{\PYGZlt{}}\PYG{o}{\PYGZlt{}}\PYG{+w}{ }\PYG{l+m+mi}{1}\PYG{+w}{ }\PYG{o}{+}\PYG{+w}{ }\PYG{l+m+mi}{1}\PYG{+w}{ }\PYG{o}{\PYGZlt{}}\PYG{o}{\PYGZlt{}}\PYG{+w}{ }\PYG{n}{std}\PYG{p}{;}\PYG{n}{endl}\PYG{p}{;}

\PYG{+w}{    }\PYG{k}{return}\PYG{+w}{ }\PYG{l+m+mi}{0}\PYG{p}{;}
\PYG{p}{\PYGZcb{}}
\end{sphinxVerbatim}

\sphinxAtStartPar
To compile a single program called \sphinxcode{\sphinxupquote{hello.cpp}} enter the command:

\begin{sphinxVerbatim}[commandchars=\\\{\}]
\PYGZdl{}\PYG{+w}{ }g++\PYG{+w}{ }\PYGZhy{}o\PYG{+w}{ }hello\PYG{+w}{ }\PYGZhy{}I../../../\PYG{+w}{ }hello.cc
\end{sphinxVerbatim}

\sphinxAtStartPar
where \sphinxcode{\sphinxupquote{\sphinxhyphen{}I../../}} is the include option indicating where the header files are located. In this case the headers are located in a folder \sphinxstylestrong{three levels above} the folder where \sphinxcode{\sphinxupquote{hello.cc}} is located and is being compiled.

\sphinxAtStartPar
To run the compiled program enter \sphinxcode{\sphinxupquote{./hello}} at the CLI

\index{index include@\spxentry{index include}}\ignorespaces 
\sphinxstepscope


\chapter{Variables Introduction}
\label{\detokenize{text/progtut/variables:variables-introduction}}\label{\detokenize{text/progtut/variables::doc}}
\sphinxAtStartPar
A \sphinxstylestrong{domain} is a collection or a range of possible values exhibiting a pattern. Objects belonging to the same domain have the same \sphinxstylestrong{type}. If an object \(x\) belongs to a domain \(D\) this is written mathematically as:
\begin{equation*}
\begin{split}
x \in D \quad \text{(Set theoretical notation)}
\end{split}
\end{equation*}
\sphinxAtStartPar
or equivalently as
\begin{equation*}
\begin{split}
x : D \quad \text{(Type theoretical notation)}
\end{split}
\end{equation*}
\sphinxAtStartPar
In C++:

\begin{sphinxVerbatim}[commandchars=\\\{\}]
\PYG{n}{D} \PYG{n}{x}\PYG{p}{;}
\end{sphinxVerbatim}

\sphinxAtStartPar
This is called the \sphinxstylestrong{declaration} of \sphinxstylestrong{variable} x of the \sphinxstylestrong{data type} D.

\sphinxAtStartPar
A (mathematical) domain can be finite, countably infinite, or uncountanbly infinite.

\begin{sphinxadmonition}{note}{Note:}
\sphinxAtStartPar
A \sphinxstylestrong{mathematical structure} is a domain equipped with distinguished \sphinxstylestrong{special elements}, \sphinxstylestrong{relations} and \sphinxstylestrong{operations}. Relations and operations are usually binary.

\sphinxAtStartPar
\sphinxstylestrong{Example}: Domain of integeres \(\mathbb{Z}\) along with the special elements \(0\) and \(1\), order relation \(<\), arithmetic operations \(+, \times\).

\sphinxAtStartPar
\(\mathbb{Z}\) \sphinxstylestrong{countably infinite}.
\end{sphinxadmonition}

\sphinxAtStartPar
computers are finite, and physical components (the arithmetic and logic unit) performing the operations are capable of holding finitely many distinct representations of objects.

\sphinxAtStartPar
Therefore the mathematical structures realized by computer hardware are not \(\langle\mathbb{Z}, 0, 1, +, \times\) and \(\langle\mathbb{R}, 0, 1, +, \times\rangle\) but their finite approximations
\(\langle\tilde{\mathbb{Z}}, 0, 1, \oplus, \otimes\rangle\) and \(\langle\mathbb{F}, 0, 1, \oplus, \otimes\rangle\), where the basic axioms are not always satisfied.

\sphinxstepscope


\section{Basic Data Tytpes in C++}
\label{\detokenize{text/progtut/datatypes:basic-data-tytpes-in-c}}\label{\detokenize{text/progtut/datatypes::doc}}
\sphinxAtStartPar
\sphinxstylestrong{basic(atomic) data types} of a programming langauge are the types directly provided by the language, as opposed to the data types defined by the programmer using the mechanisms of the langauge.

\sphinxAtStartPar
Some high\sphinxhyphen{}level programming langauges provide basic numerical data types that correspond to the ideal mathematical types \(\mathbb{R}\) and \(\mathbb{Z}\). But C++ provides only low level basic data types that are directly represented by the computer and directly operated on by the ALU. This data types are \sphinxcode{\sphinxupquote{int}}, \sphinxcode{\sphinxupquote{float}} and \sphinxcode{\sphinxupquote{double}}  correspong to \(\tilde{\mathbb{Z}}\) and \(\mathbb{F}\), respectively.


\begin{savenotes}\sphinxattablestart
\centering
\begin{tabulary}{\linewidth}[t]{|T|T|T|T|}
\hline
\sphinxstyletheadfamily 
\sphinxAtStartPar
Type
&\sphinxstyletheadfamily 
\sphinxAtStartPar
Range
&\sphinxstyletheadfamily 
\sphinxAtStartPar
Implements
&\sphinxstyletheadfamily 
\sphinxAtStartPar
Represents
\\
\hline
\sphinxAtStartPar
\sphinxcode{\sphinxupquote{int}}
&
\sphinxAtStartPar
{[}\sphinxhyphen{}2\textasciicircum{}31\textasciicircum{}, 2\textasciicircum{}31\textasciicircum{}\sphinxhyphen{}1{]}
&
\sphinxAtStartPar
IEEE int
&
\sphinxAtStartPar
\(\mathbb{Z}\)
\\
\hline
\sphinxAtStartPar
\sphinxcode{\sphinxupquote{unsigned int}}
&
\sphinxAtStartPar
{[}0, 2\textasciicircum{}32\textasciicircum{} \sphinxhyphen{} 1{]}
&
\sphinxAtStartPar
\sphinxhyphen{}
&
\sphinxAtStartPar
\(\mathbb{N}\)
\\
\hline
\sphinxAtStartPar
\sphinxcode{\sphinxupquote{float}}
&
\sphinxAtStartPar
{[}\sphinxhyphen{}3.4e38, 3.4e38{]}
&
\sphinxAtStartPar
IEEE float
&
\sphinxAtStartPar
\(\mathbb{R}\)
\\
\hline
\sphinxAtStartPar
\sphinxcode{\sphinxupquote{double}}
&
\sphinxAtStartPar
{[}\sphinxhyphen{}1.80e+308, 1.80e+308{]}
&
\sphinxAtStartPar
IEEE double
&
\sphinxAtStartPar
\(\mathbb{R}\)
\\
\hline
\sphinxAtStartPar
\sphinxcode{\sphinxupquote{char}}
&
\sphinxAtStartPar
ASCII characters
&
\sphinxAtStartPar
ASCII characters
&
\sphinxAtStartPar
letters and others
\\
\hline
\sphinxAtStartPar
\sphinxcode{\sphinxupquote{string}}
&
\sphinxAtStartPar
strings of ASCII
&
\sphinxAtStartPar
\sphinxhyphen{}
&
\sphinxAtStartPar
\sphinxhyphen{}
\\
\hline
\end{tabulary}
\par
\sphinxattableend\end{savenotes}

\sphinxAtStartPar
Table: List of Basic Data Types in C++

\sphinxstepscope


\section{Variable Declarations}
\label{\detokenize{text/progtut/declarations:variable-declarations}}\label{\detokenize{text/progtut/declarations::doc}}
\begin{sphinxuseclass}{cell}
\begin{sphinxuseclass}{tag_hide-input}
\end{sphinxuseclass}
\end{sphinxuseclass}
\sphinxAtStartPar
variables can be declared unitialized. Usually some default value like 0 is assigned to them.

\begin{sphinxuseclass}{cell}\begin{sphinxVerbatimInput}

\begin{sphinxuseclass}{cell_input}
\begin{sphinxVerbatim}[commandchars=\\\{\}]
\PYG{k+kt}{unsigned}\PYG{+w}{ }\PYG{k+kt}{int}\PYG{+w}{ }\PYG{n}{i}\PYG{p}{;}
\PYG{n}{std}\PYG{o}{:}\PYG{o}{:}\PYG{n}{cout}\PYG{+w}{ }\PYG{o}{\PYGZlt{}}\PYG{o}{\PYGZlt{}}\PYG{+w}{ }\PYG{n}{i}\PYG{+w}{ }\PYG{o}{\PYGZlt{}}\PYG{o}{\PYGZlt{}}\PYG{+w}{ }\PYG{n}{std}\PYG{o}{:}\PYG{o}{:}\PYG{n}{endl}\PYG{p}{;}
\end{sphinxVerbatim}

\end{sphinxuseclass}\end{sphinxVerbatimInput}
\begin{sphinxVerbatimOutput}

\begin{sphinxuseclass}{cell_output}
\begin{sphinxVerbatim}[commandchars=\\\{\}]
0
\end{sphinxVerbatim}

\end{sphinxuseclass}\end{sphinxVerbatimOutput}

\end{sphinxuseclass}
\sphinxAtStartPar
or initialized

\begin{sphinxuseclass}{cell}\begin{sphinxVerbatimInput}

\begin{sphinxuseclass}{cell_input}
\begin{sphinxVerbatim}[commandchars=\\\{\}]
\PYG{k+kt}{double}\PYG{+w}{ }\PYG{n+nf}{x}\PYG{p}{(}\PYG{l+m+mf}{3.14}\PYG{p}{)}\PYG{p}{;}\PYG{+w}{ }
\PYG{k+kt}{float}\PYG{+w}{ }\PYG{n+nf}{y}\PYG{p}{(}\PYG{l+m+mf}{1.0}\PYG{p}{)}\PYG{p}{;}
\PYG{k+kt}{short}\PYG{+w}{ }\PYG{n+nf}{j}\PYG{p}{(}\PYG{l+m+mi}{3}\PYG{p}{)}\PYG{p}{;}
\PYG{n}{std}\PYG{o}{:}\PYG{o}{:}\PYG{n}{cout}\PYG{+w}{ }\PYG{o}{\PYGZlt{}}\PYG{o}{\PYGZlt{}}\PYG{+w}{ }\PYG{n}{x}\PYG{+w}{ }\PYG{o}{\PYGZlt{}}\PYG{o}{\PYGZlt{}}\PYG{+w}{ }\PYG{l+s}{\PYGZdq{}}\PYG{l+s}{ }\PYG{l+s}{\PYGZdq{}}\PYG{+w}{ }\PYG{o}{\PYGZlt{}}\PYG{o}{\PYGZlt{}}\PYG{+w}{ }\PYG{n}{y}\PYG{+w}{ }\PYG{o}{\PYGZlt{}}\PYG{o}{\PYGZlt{}}\PYG{+w}{ }\PYG{l+s}{\PYGZdq{}}\PYG{l+s}{ }\PYG{l+s}{\PYGZdq{}}\PYG{+w}{ }\PYG{o}{\PYGZlt{}}\PYG{o}{\PYGZlt{}}\PYG{+w}{ }\PYG{n}{j}\PYG{+w}{ }\PYG{o}{\PYGZlt{}}\PYG{o}{\PYGZlt{}}\PYG{+w}{ }\PYG{n}{std}\PYG{o}{:}\PYG{o}{:}\PYG{n}{endl}\PYG{p}{;}
\end{sphinxVerbatim}

\end{sphinxuseclass}\end{sphinxVerbatimInput}
\begin{sphinxVerbatimOutput}

\begin{sphinxuseclass}{cell_output}
\begin{sphinxVerbatim}[commandchars=\\\{\}]
3.14 1 3
\end{sphinxVerbatim}

\end{sphinxuseclass}\end{sphinxVerbatimOutput}

\end{sphinxuseclass}
\sphinxAtStartPar
full program text:

\begin{sphinxVerbatim}[commandchars=\\\{\}]
\PYG{c+cp}{\PYGZsh{}}\PYG{c+cp}{include}\PYG{+w}{ }\PYG{c+cpf}{\PYGZlt{}iostream\PYGZgt{}}

\PYG{k+kt}{int}\PYG{+w}{ }\PYG{n+nf}{main}\PYG{p}{(}\PYG{p}{)}
\PYG{p}{\PYGZob{}}
\PYG{+w}{    }\PYG{k+kt}{unsigned}\PYG{+w}{ }\PYG{k+kt}{int}\PYG{+w}{ }\PYG{n}{i}\PYG{p}{;}
\PYG{+w}{    }\PYG{k+kt}{double}\PYG{+w}{ }\PYG{n}{x}\PYG{p}{(}\PYG{l+m+mf}{3.14}\PYG{p}{)}\PYG{p}{;}\PYG{+w}{ }
\PYG{+w}{    }\PYG{k+kt}{float}\PYG{+w}{ }\PYG{n}{y}\PYG{p}{(}\PYG{l+m+mf}{1.0}\PYG{p}{)}\PYG{p}{;}
\PYG{+w}{    }\PYG{k+kt}{short}\PYG{+w}{ }\PYG{n}{j}\PYG{p}{(}\PYG{l+m+mi}{3}\PYG{p}{)}\PYG{p}{;}
\PYG{+w}{    }\PYG{n}{std}\PYG{o}{:}\PYG{o}{:}\PYG{n}{cout}\PYG{+w}{ }\PYG{o}{\PYGZlt{}}\PYG{o}{\PYGZlt{}}\PYG{+w}{ }\PYG{n}{x}\PYG{+w}{ }\PYG{o}{\PYGZlt{}}\PYG{o}{\PYGZlt{}}\PYG{+w}{ }\PYG{l+s}{\PYGZdq{}}\PYG{l+s}{ }\PYG{l+s}{\PYGZdq{}}\PYG{+w}{ }\PYG{o}{\PYGZlt{}}\PYG{o}{\PYGZlt{}}\PYG{+w}{ }\PYG{n}{y}\PYG{+w}{ }\PYG{o}{\PYGZlt{}}\PYG{o}{\PYGZlt{}}\PYG{+w}{ }\PYG{l+s}{\PYGZdq{}}\PYG{l+s}{ }\PYG{l+s}{\PYGZdq{}}\PYG{+w}{ }\PYG{o}{\PYGZlt{}}\PYG{o}{\PYGZlt{}}\PYG{+w}{ }\PYG{n}{j}\PYG{+w}{ }\PYG{o}{\PYGZlt{}}\PYG{o}{\PYGZlt{}}\PYG{+w}{ }\PYG{n}{std}\PYG{o}{:}\PYG{o}{:}\PYG{n}{endl}\PYG{p}{;}

\PYG{p}{\PYGZcb{}}
\end{sphinxVerbatim}

\sphinxstepscope


\section{Statements and Expressions}
\label{\detokenize{text/progtut/statements:statements-and-expressions}}\label{\detokenize{text/progtut/statements::doc}}

\subsection{Statements}
\label{\detokenize{text/progtut/statements:statements}}
\sphinxAtStartPar
An object of a certain type can take different values during its existense. This transformation of values is called \sphinxstylestrong{change of state} of the object.

\sphinxAtStartPar
Computers can change the sate of an object residing in memory. This is called an \sphinxstylestrong{action}. The \sphinxstylestrong{instructions} to peform actions in a given programming language are called \sphinxstylestrong{statements}.

\sphinxAtStartPar
The change of variables state caused by a statement is said to be the \sphinxstylestrong{effect} of the statement.


\subsection{Expressions}
\label{\detokenize{text/progtut/statements:expressions}}
\sphinxAtStartPar
\sphinxstylestrong{Expressions}  do not transform the state of the variables, but denote \sphinxstylestrong{values}.

\sphinxAtStartPar
Expressions are formed according to the rules of some formal notation (like the language of arithemtic)

\sphinxAtStartPar
Some expressions: \sphinxcode{\sphinxupquote{1 \sphinxhyphen{} (3 / 6)}}, \sphinxcode{\sphinxupquote{1 + x * (y \% 10)}}

\sphinxAtStartPar
Thus:
\begin{itemize}
\item {} 
\sphinxAtStartPar
expressions \sphinxstylestrong{denote values},

\item {} 
\sphinxAtStartPar
statements \sphinxstylestrong{have effects}

\end{itemize}

\begin{sphinxadmonition}{note}{Note:}
\sphinxAtStartPar
Understanding this dichotomy between statements and expessions is fundamental.
\end{sphinxadmonition}


\subsection{Assignment}
\label{\detokenize{text/progtut/statements:assignment}}
\sphinxAtStartPar
The most basic statement is the \sphinxstylestrong{assignment}. It has the \sphinxstylestrong{effect} of updating the value of the variable to the value denoted by the \sphinxstylestrong{expression} on the right\sphinxhyphen{}hand side of the assignment operator. In C++:

\begin{sphinxuseclass}{cell}\begin{sphinxVerbatimInput}

\begin{sphinxuseclass}{cell_input}
\begin{sphinxVerbatim}[commandchars=\\\{\}]
\PYG{c+cp}{\PYGZsh{}}\PYG{c+cp}{include}\PYG{+w}{ }\PYG{c+cpf}{\PYGZlt{}iostream\PYGZgt{}}

\PYG{k+kt}{int}\PYG{+w}{ }\PYG{n+nf}{x}\PYG{p}{(}\PYG{l+m+mi}{1}\PYG{p}{)}\PYG{p}{;}\PYG{+w}{  }\PYG{c+c1}{//declare \PYGZam{} initialize x}
\PYG{k+kt}{int}\PYG{+w}{ }\PYG{n}{y}\PYG{+w}{ }\PYG{o}{=}\PYG{+w}{ }\PYG{n}{x}\PYG{p}{;}\PYG{+w}{ }\PYG{c+c1}{//declare y and assign the value denoted by the expression \PYGZdq{}x\PYGZdq{} to y. }

\PYG{n}{std}\PYG{o}{:}\PYG{o}{:}\PYG{n}{cout}\PYG{+w}{ }\PYG{o}{\PYGZlt{}}\PYG{o}{\PYGZlt{}}\PYG{+w}{ }\PYG{n}{x}\PYG{+w}{ }\PYG{o}{\PYGZlt{}}\PYG{o}{\PYGZlt{}}\PYG{+w}{ }\PYG{l+s}{\PYGZdq{}}\PYG{l+s}{ }\PYG{l+s}{\PYGZdq{}}\PYG{+w}{ }\PYG{o}{\PYGZlt{}}\PYG{o}{\PYGZlt{}}\PYG{+w}{ }\PYG{n}{y}\PYG{+w}{ }\PYG{o}{\PYGZlt{}}\PYG{o}{\PYGZlt{}}\PYG{+w}{ }\PYG{l+s}{\PYGZdq{}}\PYG{l+s}{ }\PYG{l+s}{\PYGZdq{}}\PYG{+w}{ }\PYG{o}{\PYGZlt{}}\PYG{o}{\PYGZlt{}}\PYG{+w}{ }\PYG{n}{std}\PYG{o}{:}\PYG{o}{:}\PYG{n}{endl}\PYG{p}{;}
\end{sphinxVerbatim}

\end{sphinxuseclass}\end{sphinxVerbatimInput}
\begin{sphinxVerbatimOutput}

\begin{sphinxuseclass}{cell_output}
\begin{sphinxVerbatim}[commandchars=\\\{\}]
1 1 
\end{sphinxVerbatim}

\end{sphinxuseclass}\end{sphinxVerbatimOutput}

\end{sphinxuseclass}
\sphinxAtStartPar
The expression on the right hand side of the assignment can contain the variable being updated. Expression is evaluated before assignment:

\begin{sphinxuseclass}{cell}\begin{sphinxVerbatimInput}

\begin{sphinxuseclass}{cell_input}
\begin{sphinxVerbatim}[commandchars=\\\{\}]
\PYG{n}{y}\PYG{+w}{ }\PYG{o}{=}\PYG{+w}{ }\PYG{p}{(}\PYG{n}{y}\PYG{+w}{ }\PYG{o}{*}\PYG{+w}{ }\PYG{l+m+mi}{3}\PYG{p}{)}\PYG{+w}{ }\PYG{o}{+}\PYG{+w}{ }\PYG{n}{x}\PYG{p}{;}\PYG{+w}{ }\PYG{c+c1}{//assign the value denoted by (y * 3) + x to y. }
\PYG{n}{std}\PYG{o}{:}\PYG{o}{:}\PYG{n}{cout}\PYG{+w}{ }\PYG{o}{\PYGZlt{}}\PYG{o}{\PYGZlt{}}\PYG{+w}{ }\PYG{n}{y}\PYG{+w}{ }\PYG{o}{\PYGZlt{}}\PYG{o}{\PYGZlt{}}\PYG{+w}{ }\PYG{n}{std}\PYG{o}{:}\PYG{o}{:}\PYG{n}{endl}\PYG{p}{;}
\end{sphinxVerbatim}

\end{sphinxuseclass}\end{sphinxVerbatimInput}
\begin{sphinxVerbatimOutput}

\begin{sphinxuseclass}{cell_output}
\begin{sphinxVerbatim}[commandchars=\\\{\}]
4
\end{sphinxVerbatim}

\end{sphinxuseclass}\end{sphinxVerbatimOutput}

\end{sphinxuseclass}

\subsubsection{Side Effects}
\label{\detokenize{text/progtut/statements:side-effects}}
\sphinxAtStartPar
In some programming languages like Pascal the world of statements and the world of expressions are completely distinct. A given syntactical entity is either a statement or an expression.

\sphinxAtStartPar
But in C/C++ a syntactical entity can be both a statement (have an effect) and an expression (denote a value).

\sphinxAtStartPar
The action that such an expression performs is called the \sphinxstylestrong{side effect} of the expression.

\sphinxAtStartPar
\sphinxstylestrong{Example}: \sphinxcode{\sphinxupquote{j\sphinxhyphen{}\sphinxhyphen{}}} is both a statement and an expression. Its effect is to assign to the variable \sphinxcode{\sphinxupquote{j}} the value of the expression \sphinxcode{\sphinxupquote{j \sphinxhyphen{} 1}}. I.e. it is equivalent to \sphinxcode{\sphinxupquote{j = j \sphinxhyphen{} 1}}. As an expression it denotes the value \sphinxcode{\sphinxupquote{j}} before assignment.

\begin{sphinxuseclass}{cell}\begin{sphinxVerbatimInput}

\begin{sphinxuseclass}{cell_input}
\begin{sphinxVerbatim}[commandchars=\\\{\}]
\PYG{k+kt}{int}\PYG{+w}{ }\PYG{n}{j}\PYG{+w}{ }\PYG{o}{=}\PYG{+w}{ }\PYG{l+m+mi}{10}\PYG{p}{;}

\PYG{n}{std}\PYG{o}{:}\PYG{o}{:}\PYG{n}{cout}\PYG{+w}{ }\PYG{o}{\PYGZlt{}}\PYG{o}{\PYGZlt{}}\PYG{+w}{ }\PYG{n}{j}\PYG{o}{\PYGZhy{}}\PYG{o}{\PYGZhy{}}\PYG{+w}{ }\PYG{o}{\PYGZlt{}}\PYG{o}{\PYGZlt{}}\PYG{+w}{ }\PYG{n}{std}\PYG{o}{:}\PYG{o}{:}\PYG{n}{endl}\PYG{p}{;}\PYG{+w}{ }\PYG{c+c1}{//side effect: decrement j. }
\PYG{n}{std}\PYG{o}{:}\PYG{o}{:}\PYG{+w}{ }\PYG{n}{cout}\PYG{+w}{ }\PYG{o}{\PYGZlt{}}\PYG{o}{\PYGZlt{}}\PYG{+w}{ }\PYG{n}{j}\PYG{+w}{ }\PYG{o}{\PYGZlt{}}\PYG{o}{\PYGZlt{}}\PYG{+w}{ }\PYG{n}{std}\PYG{o}{:}\PYG{o}{:}\PYG{n}{endl}\PYG{p}{;}
\end{sphinxVerbatim}

\end{sphinxuseclass}\end{sphinxVerbatimInput}
\begin{sphinxVerbatimOutput}

\begin{sphinxuseclass}{cell_output}
\begin{sphinxVerbatim}[commandchars=\\\{\}]
10
\end{sphinxVerbatim}

\begin{sphinxVerbatim}[commandchars=\\\{\}]
9
\end{sphinxVerbatim}

\end{sphinxuseclass}\end{sphinxVerbatimOutput}

\end{sphinxuseclass}
\sphinxAtStartPar
Havinhg expression with side effects is academically less clean, but it practical benefits as it allows shorter programming constructs and succinct idioms like:

\begin{sphinxuseclass}{cell}\begin{sphinxVerbatimInput}

\begin{sphinxuseclass}{cell_input}
\begin{sphinxVerbatim}[commandchars=\\\{\}]
\PYG{k+kt}{int}\PYG{+w}{ }\PYG{n}{i}\PYG{+w}{ }\PYG{o}{=}\PYG{+w}{ }\PYG{l+m+mi}{10}\PYG{p}{;}
\PYG{k}{while}\PYG{+w}{ }\PYG{p}{(}\PYG{n}{i}\PYG{o}{\PYGZhy{}}\PYG{o}{\PYGZhy{}}\PYG{p}{)}\PYG{p}{\PYGZob{}}
\PYG{+w}{    }\PYG{n}{std}\PYG{o}{:}\PYG{o}{:}\PYG{n}{cout}\PYG{+w}{ }\PYG{o}{\PYGZlt{}}\PYG{o}{\PYGZlt{}}\PYG{+w}{ }\PYG{n}{i}\PYG{+w}{ }\PYG{o}{\PYGZlt{}}\PYG{o}{\PYGZlt{}}\PYG{+w}{ }\PYG{l+s}{\PYGZdq{}}\PYG{l+s}{ }\PYG{l+s}{\PYGZdq{}}\PYG{p}{;}\PYG{+w}{ }
\PYG{p}{\PYGZcb{}}
\end{sphinxVerbatim}

\end{sphinxuseclass}\end{sphinxVerbatimInput}
\begin{sphinxVerbatimOutput}

\begin{sphinxuseclass}{cell_output}
\begin{sphinxVerbatim}[commandchars=\\\{\}]
9 8 7 6 5 4 3 2 1 0 
\end{sphinxVerbatim}

\end{sphinxuseclass}\end{sphinxVerbatimOutput}

\end{sphinxuseclass}
\sphinxstepscope


\part{Numerical Methods}

\sphinxstepscope


\chapter{Introduction}
\label{\detokenize{text/num/intro:introduction}}\label{\detokenize{text/num/intro::doc}}
\sphinxAtStartPar
Introduction to Numerical Methods Notes.







\renewcommand{\indexname}{Index}
\printindex
\end{document}